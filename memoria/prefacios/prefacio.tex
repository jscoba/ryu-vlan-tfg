\chapter*{}
%\thispagestyle{empty}
%\cleardoublepage

%\thispagestyle{empty}




\cleardoublepage
\thispagestyle{empty}

\begin{center}
{\large\bfseries Diseño e implementación de una red inteligente autoconfigurable}\\
\end{center}
\begin{center}
Javier Sáez de la Coba\\
\end{center}

%\vspace{0.7cm}
\noindent{\textbf{Palabras clave}: redes, Software Defined Networks, OpenFlow, mininet, Ryu, VLAN}\\

\vspace{0.7cm}
\noindent{\textbf{Resumen}}\\

%Poner aquí el resumen.\todo{Pon: un párrafo para hablar de redes SDN como herramienta habilitadora de rdes futuras, como 5G (hay un ejemplo de google y de otra empresa, que usan openflow). Luego, otro párrafo comentando que las redes van hacia la autoconfiguración. Luego comentas que en este proyecto has abordado el diseño e implementaión de una red autoconfigurable basada en SDN para un escenario tal.... Luego, pon qué cosas hace tu red. Y termina con otro párrafo que diga los resultados de la evaluación (si has podido hacerla. Si no, el resutlado serían als funciones que has desarrollado.}

Las redes definidas por software han supuesto una revolución en el mundo de las redes de ordenadores, su versatilidad y potencia las hacen herramientas habilitadoras de redes de última generación como 5G o grandes redes como las de Google. Además, las redes de ordenadores, motivado en gran parte por la aparición de las Infraestructuras como servicio (IaaS), tienden cada vez más a la configuración automática.

En este proyecto vamos a diseñar e implementar una red autoconfigurable definida por software utilizando una implementación de Openflow y un software de emulación de redes para un escenario de oficinas. La red será capaz de hacer una asignación dinámica de VLAN's basada en las direcciones MAC de los dispositivos finales así como detectar y corregir los bucles y enlaces caídos que pudiera haber en la misma.


\cleardoublepage


\thispagestyle{empty}


\begin{center}
{\large\bfseries Design and implementation of a intelligent and autoconfigurable computer network: Software Defined Networks using OpenFlow and Mininet-wifi}\\
\end{center}
\begin{center}
Javier Sáez de la Coba\\
\end{center}

%\vspace{0.7cm}
\noindent{\textbf{Keywords}: networking, Software Defined Networks, OpenFlow, mininet, Ryu, VLAN}\\

\vspace{0.7cm}
\noindent{\textbf{Abstract}}\\

Software Defined Networking has become a revolution in the world of networking. Its capacity and adaptability has become an enabling tool for the next-generation networks such as 5G or large networks as Google's. Moreover, Computer networks, largely motivated by the emergence of Infrastructure as a Service (IaaS), are increasingly tending towards automatic configuration.

In this project we are going to design and implement an autoconfigurable computer network using an Openflow implementation and a network simulator software for an office building case scenario. The network will be able to dinamically determine and assign the VLAN network of a host depending on its MAC address. It will also detect and correct the posible looped or broken links to maintain correct operation.



