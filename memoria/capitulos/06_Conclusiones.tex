\chapter{Conclusiones}

En este trabajo hemos logrado implementar con éxito una red autoconfigurable mediante Ryu. La red es capaz de detectar los dispositivos que tiene conectados y adaptar sus tablas de encaminamiento a dichos dispositivos. Además es capaz de asignar de forma dinámica una VLAN a cada nodo en el momento de conectarse basado en su dirección MAC sin la necesidad de intervención externa.

La red también es redundante en enlaces y es capaz de recuperar la conectividad de manera automática en caso de conexión de una caída de enlace.

Tras haber hecho un estudio exhaustivo de la arquitectura de redes definidas por software y haber desarrollado una podemos afirmar que, aunque la carga de trabajo a la hora de instalar la red sea grande esta facilita enormemente la administración y el mantenimiento de la misma, con su correspondiente mejora en eficacia y reducción de costes.

Como trabajo futuro se deja la inclusión de un sistema de Calidad de Servicio basado en las colas de prioridad de las que disponen los switches OpenFlow o la inclusión de un protocolo de \textit{link aggregation}.

En definitiva tenemos un proyecto software que puede ser usado, con algunos ligeros ajustes en entornos SDN reales.